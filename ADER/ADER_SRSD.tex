\documentclass[12pt]{article}
\usepackage[english]{babel}
\renewcommand{\familydefault}{\rmdefault}
\usepackage[left=1in,right=1in,top=1in,bottom=1in]{geometry}
\usepackage[version=3]{mhchem}
\title{Software Requirements Specifications Document for the Advanced Depletion Extension for Reprocessing (ADER) SERPENT-2 Modification \\ {\small \date{\today}}}
\author{Daniel Wooten}

\begin{document}

\maketitle

\section{Introduction}
The ADER modification to the SERPENT2 code shall enable the code user, via
inputs incorporated into the standard SERPENT2 input, to model the burnup of materials which
undergo discrete and/or continous modifications to their composition throughout
the simulation. It is the goal of this modification to be as general as
possible. Many of the changes to material composition which can be modeled
with ADER do not represent a current physical reality. Rather than reflect
the limits of today's technology the goal of this modification is to allow
the user to model arbitrary removal, addition, and chemistry control scenarios.
\section{Requirements}
\subsection{Input}
The inputs required for ADER shall be part of the standard SERPENT2 input. No
user inputs shall require recompilation of the code.
Additionally, ADER will not seek to enforce "resonable" input. It
will be entirely possible for a user to provide input which will prevent
the simulation from running; however, in such cases it is the goal of ADER
to not only error out of the simulation but to provide an insightful and
hopefully helpful error message which may provide clarity to the user as to
the nature of their input which caused the simulation to error out. A key
philosophy of this modification will be to crash easily but with clarity.
\subsection{Continuous Removal}
ADER shall allow for the modeling of continuous and discreet removal of isotopes,
elements, and groups of elements. Continuous removal may be modeled as both
a fractional extraction process, similar to radioactive decay, and a continuous
mass removal process. Specific removal constants for isotopes may be specified
on an isotope by isotope basis, or, they may be specified as a desired fraction
of an overall elemental removal.
For example, while
Nb may have a removal fraction of 0.1 per day, it may be specified that
\ce{^{93}_{41}Nb} has a removal fraction of 0.5 per day. \ce{^{93}_{41}Nb}
would then be removed at a faster rate than other Nb isotopes and its removal
would not contribute to the 0.1 fraction for all Nb. However, if 
\ce{^{93}_{41}Nb} was specified as 20\% of the removed Nb then 20\%,
if possible, of the Nb rmeoved at its fractional rate of 0.1 per day would be
composed of \ce{^{93}_{41}Nb}.  
\subsection{Reactive Feed}
ADER shall allow for the modeling of reactivity control via addition or
removal, continuous or discreet, of an arbitrary number of groups of elements.
These groups can be
specified to hold a certain collection of elements which themselves
may be specified to be of a specific isotopic composition. Additionally,
groups may be flagged as only specifying elements to be added, removed, or both.The amount of reactive feed or removal is determined in conjunction with
chemistry control at each burnup sub-step and relies on a k_{\infty} estimator
as well as a leakage estimator corrector value.
\subsection{Composition Control}
ADER shall allow for the specification of groups of elements, which may also
have isotopic compositions specified, within a material
to be "controlled" for their fraction of that material's composition. For
example, a group could be specified containing Li, F, Na, and K in a given
ratio, and could
be specified to constitue 80\% of the given material's composition
throughout the simulation. Furthermore, ratios with respect to other groups
of elements could be stipulated. 
All compositions and ratios, from the fraction of a
 group in a material to the fraction of an isotope in an element,
may be specified as a range with a minimum
and a maximum bound, as well as one child per parent being specified to fill
the "remainder" of the parent.
A given element or isotope may be specified to be a part
of more than 1 group.\par
Each material may be given a set of feed and removal
groups. Each of these groups may have the same level of detail as the
material composition groups; ratios, ranges, and isotopic specifications.
\subsection{Valance Control}
ADER shall provide means through which the REDOX potential of a material may
be controlled. While this aspect of ADER does not directly model the REDOX
potential evolution of a material it allows the user to set the oxidation
state of elements in the material as well as provide a means by which ADER
can maintain an average oxidation state in the material.
The required inputs for valance control are an average oxidation state
target value and a list of all elements, their average oxidation state
in the material, and an optional weighting factor for each element.
Additionally, feed and removal groups for valance control may be
specified. \par
During the composition control stage, if valance control is enabled,
ADER will impose as an addiitional constraint on the material's composition
that said composition satisfy the valance target using the given oxidation
state list.
\section{Additional Components}
Because the composition control operations involve optimizing a linear set
of equations it is the current intention to employ the Coin-OR LP solver
library as the work horse for finding such an optimization. This library
is written in C++ and is licenced under the Common Public Licence which
allows for its incoorporation into other software packages using
more restrictive licences. 
\end{document}
