\documentclass[12pt]{article}
\usepackage[english]{babel}
\renewcommand{\familydefault}{\rmdefault}
\usepackage[left=1in,right=1in,top=1in,bottom=1in]{geometry}
\usepackage[version=3]{mhchem}
\title{Software Requirements Specifications Document for the Advanced Depletion Extension for Reprocessing (ADER) SERPENT-2 Modification \\ {\small \date{\today}}}
\author{Daniel Wooten}

\begin{document}

\maketitle

\section{Introduction}
The ADER modification to the SERPENT2 code shall enable the code user, via
inputs to the standard SERPENT2 input, to model the burnup of materials which
undergo discrete and/or continous modifications to their composition throughout
the simulation.
\section{Requirements}
\subsection{Input}
The inputs required for ADER shall be part of the standard SERPENT2 input. No
user inputs shall require recompilation of the code.
\subsection{Continuous Removal}
ADER shall allow for the modeling of continuous fractional removal of isotopes,
elements, and groups of elements with the specification of
fractional removal constants that shall not vary over the simulation. An
exception to this is that for isotopes, if they are specified, they may be given
either their own fractional removal rates that will override the elemental
fractional removal rate or they may be given as a fraction of the total
elemental removal. 
For example, while
Nb may have a removal fraction of 0.1 per day, it may be specified that
\ce{^{93}_{41}Nb} has a removal fraction of 0.5 per day. \ce{^{93}_{41}Nb}
would then be removed at a faster rate than other Nb isotopes and its removal
would not contribute to the 0.1 fraction for all Nb. However, if 
\ce{^{93}_{41}Nb} was specified as 20\% of the removed Nb then 20\%,
if possible, of the Nb rmeoved at its fractional rate of 0.1 per day would be
composed of \ce{^{93}_{41}Nb}.  
\end{document}
