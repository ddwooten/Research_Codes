\documentclass[12pt]{article}
\usepackage[english]{babel}
\renewcommand{\familydefault}{\rmdefault}
\usepackage[left=1in,right=1in,top=1in,bottom=1in]{geometry}
\usepackage[version=3]{mhchem}
\title{Software Requirements Specifications Document for the Advanced Depletion Extension for Reprocessing (ADER) SERPENT-2 Modification \\ {\small \date{\today}}}
\author{Daniel Wooten}

\begin{document}

\maketitle

\section{Introduction}
The ADER modification to the SERPENT2 code shall enable the code user, via
inputs to the standard SERPENT2 input, to model the burnup of materials which
undergo discrete and/or continous modifications to their composition throughout
the simulation. It is the goal of this modification to be as general as
possible. Many of the changes to material composition which may be modeled
with ADER do not represent a current physical reality. Rather than reflect
the limits of today's technology the goal of this modification is to allow
the user to model arbitrary removal, addition, and chemistry control
scenarios. Additionally, ADER will not seek to enforce "resonable" input. It
will be entirely possible for a user to provide input which will prevent
the simulation from running; however, in such cases it is the goal of ADER
to not only error out of the simulation but to provide an insightful and
hopefully helpful error message which may provide clarity to the user as to
the nature of their input which caused the simulation to error out. A key
philosophy of this modification will be to crash easily but with clarity.
\section{Requirements}
\subsection{Input}
The inputs required for ADER shall be part of the standard SERPENT2 input. No
user inputs shall require recompilation of the code.
\subsection{Continuous Removal}
ADER shall allow for the modeling of continuous fractional removal of isotopes,
elements, and groups of elements with the specification of
fractional removal constants that shall not vary over the simulation. An
exception to this is that for isotopes, if they are specified, they may be given
either their own fractional removal rates that will override the elemental
fractional removal rate or they may be given as a fraction of the total
elemental removal. 
For example, while
Nb may have a removal fraction of 0.1 per day, it may be specified that
\ce{^{93}_{41}Nb} has a removal fraction of 0.5 per day. \ce{^{93}_{41}Nb}
would then be removed at a faster rate than other Nb isotopes and its removal
would not contribute to the 0.1 fraction for all Nb. However, if 
\ce{^{93}_{41}Nb} was specified as 20\% of the removed Nb then 20\%,
if possible, of the Nb rmeoved at its fractional rate of 0.1 per day would be
composed of \ce{^{93}_{41}Nb}.  
\subsection{Reactive Feed}
ADER shall allow for the modeling of reactivity control via addition or
removal of two groups of elements. One group should be supplied such that its
addition would increase reactivity while the other group should be supplied
such that its addition would decrease reactivity. These groups can be
specified to hold a certain collection of elements which themselves
may be specified to be of a specific isotopic composition. Removal or addition
of either group is made possible by specifiying the elemental composition
for the removal or addition component of both the "increasing" and "decreasing"
reactivity groups. If both addition and removal options are specified for
both the "increasing" and "decreasing" groups then the net change in the number
of mols of material adjusted by the reactive feed operation will be sought to
be zero. For example, if reactivity increase is desired feed from the addition
portion of the "increasing" group would be added while material from the
removal portion of the "decreasing" group would be removed in an equal amount.
\subsection{Composition Control}
ADER shall allow for the specification of groups of elements within a material
to be "controlled" for their fraction of that material's composition. For
example, a group could be specified containing Li, F, Na, and K in a given
ratio, and could
be specified to constitue 80\% of the given material's composition
throughout the simulation. Additionally, each element in a group may be given
an isotopic composition. All compositions, from the fraction of a group in a
material
to the fraction of an isotope in an element,
may be specified as a range with a minimum
and a maximum bound. A given element or isotope may be specified to be a part
of more than 1 group. That component's total amount in the material will be
the sum of its presence in all the groups to which it belongs. \par
Each group in a material may be given a corresponding feed and removal
vector. Each of these is built in the same fashion as the groups themselves
with elements having a fraction of the composition and the option to specify
isotopic concentrations for elements. Even if a group is specified and has a
non-zero minimum composition target it is not required to have a feed vector.
If the group is depeleted below its minimum composition target the simulation
will error out. The same behavior is seen for the removal vector. Additionally,
removal for a group may be done in two ways if it is specified to have removal.
For the first method a removal vector may be specified from which batch
removal of the
constituents of a group, found in its removal vector, will occur at burnup 
steps to bring the group within its composition target. The second method
will rely on the continuous removal constants defined for the continuous
removal function. In this second method ADER will solve for the volume fraction
of the material to be reprocessed, per day, in order for the groups to meet
their composition targets.
\subsection{Valance Control}
ADER shall provide means through which the REDOX potential of a material may
be controlled. While this aspect of ADER does not directly model the REDOX
potential evolution of a material it allows the user to set the oxidation
state of elements in the material as well as provide a means by which ADER
can maintain an average oxidation state in the material. This method takes up
to five inputs. The first input is required if the other possible
 four inputs are
desired to be used. The first input is a list of all elements and their
average oxidation state in the material.
 The following four inputs are four lists
of elements and their fraction of the composition of the list they are in.
These elements may, like the other inputs, be given isotopic compositions as
well.
These four lists are broken into two groups, elements with a positive average
valance state and elements with a negative average valance state. Each of these
groups has two lists, one of elements that can be added, one of elements that
can be removed. \par
During the composition control stage, if valance control is enabled,
ADER will calculate the average valance of the material. 
If the material
requires the addition of positive valance and both an addition list for
positive valance elements is specified and a removal list for negative valance
elements is specified 

\end{document}
